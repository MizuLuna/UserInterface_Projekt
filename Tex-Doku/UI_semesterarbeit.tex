%-----------------------------------------------------------------------------
% Schriftgröße, Layout, Papierformat, Art des Dokumentes
%-----------------------------------------------------------------------------
\documentclass[12pt,					% Grundschriftgöße
							 oneside,			% einseitiges Dokument
							 a4paper,			% Papiergröße
							 halfparskip,		% Einzug bei einem Absatz
							 liststotoc,			% Verzeichniss (Abbildungen erc.) in das Inhaltverzeichnis
							 bibtotoc,			% Literaturverzeichnis ins Inhaltverzeichnis
							 fleqn,				% Mathematische Formeln linksbündig darstellen
							 pointlessnumbers]	% Punkt am Ende der Nummerierung des Inhaltsverzeichnisses entfernen
							 {scrreprt}

%-----------------------------------------------------------------------------
% Konstanten festlegen
%-----------------------------------------------------------------------------
\newcommand{\VerfasserA}{Nicole Goldmann}
\newcommand{\GeburtstagA}{20. August 1997}
\newcommand{\GeburtsortA}{Herbolzheim}
\newcommand{\VerfasserB}{Stefanie Weidemann}
\newcommand{\GeburtstagB}{25. August 1995}
\newcommand{\GeburtsortB}{Anklam}
\newcommand{\Titel}{Webapp für Studiende des Studienganges AIMMT}
\newcommand{\Betreuer}{Prof. Dr.-Ing. Herbert Litschke}
\newcommand{\blankpage}{
	\newpage
	\thispagestyle{empty}
	\mbox{}
	\newpage
}

%-----------------------------------------------------------------------------
% verwendete Pakete
%-----------------------------------------------------------------------------
\usepackage[T1]{fontenc}				% Wahl des Fonts, bzw. der Kodierung
\usepackage[utf8]{inputenc}			% Zeichkodierung , Umlaute erlauben
\usepackage[english,ngerman]{babel}		% neue deutsche Rechtschreibung verwenden
\usepackage{graphicx}					% erm�glicht das Einbinden von Grafiken, sehr wichtig!
\usepackage{fancyhdr}					% f�r formatierte Kopf- und Fu�zeilen
\usepackage{setspace}					% Package zum Kontrollieren von Leerr�umen
\usepackage{subfigure}					% erweiterte Darstellung von Bildern
\usepackage{listings}					% M�glickeit zum Anzeigen von Quelltexten
\usepackage{color,moreverb}				% Farben
\usepackage{lmodern}					% bietet neuere Schriften, sieht besser aus im Acrobat Reader
\usepackage{amsmath,amssymb}			% erweiteter Formelsatz und zus�tzliche Mathe-Symbole
\usepackage{booktabs}					% professionelle, typographisch richtige Tabellen
\usepackage{cite}						% f�r LibTex
%\usepackage{shortvrb}					% f�r Quellcode mit \begin{verbatim}
\usepackage[binary-units=true]{siunitx}	% Darstellung von Si-Einheiten
\usepackage{enumitem}					% custom itemiziation
\usepackage{textcomp}
\usepackage{url}
%-----------------------------------------------------------------------------
% Fu�notennummerierung nicht f�r jedes kapitel zur�cksetzen
%-----------------------------------------------------------------------------
\usepackage{chngcntr}
\counterwithout{footnote}{chapter}

%-----------------------------------------------------------------------------
% Einstellungen der Seitenr�nder
%-----------------------------------------------------------------------------
\usepackage[left=3cm,						% linker Rand
						right=3cm,			% rechter Rand
						top=1.5cm,			% oberer Rand
						bottom=1.5cm,		% unterer Rand
						includeheadfoot,	% bezieht die Kopf- und Fu�zeile mit ein
						bindingoffset=0cm]	% Bundsteg
						{geometry}

%-----------------------------------------------------------------------------
% Daten f�r die Titel des Artikels
%-----------------------------------------------------------------------------
\title{Semesterarbeit}
\author{\VerfasserA, VerfasserB}
\date{\today{}}

%-----------------------------------------------------------------------------
% Metadaten in pdf einf�gen
%-----------------------------------------------------------------------------
\usepackage[pdftex,
						pdfauthor={\VerfasserA, VerfasserA},					% Name des Autors
						pdftitle={\Titel},										% Name der Arbeit
						pdfcreator={MiKTeX, LaTeX with hyperref and KOMA-Script},	% Von was erzeugt
						%pdfsubject={Praktikumsbericht},							% Was f�r eine Arbeit ist es
						pdfkeywords={\Titel},
						plainpages=false,
						hypertexnames=false,
						pdfpagelabels]{hyperref}

%-----------------------------------------------------------------------------
% Schriftarten anpassen
%-----------------------------------------------------------------------------
\setkomafont{sectioning}{\rmfamily\bfseries}			% Titelzeilen
\setkomafont{caption}{\small}							% Schrift f�r Caption
\setkomafont{captionlabel}{\sffamily\bfseries\small}	% Schrift f�r 'Abbildung'
\setkomafont{chapterentry}{\small\bfseries}				% Schrift f�r Inhaltsverzeichnis
\setkomafont{chapter}{\large\bfseries}					% Schrift f�r Kapitel
\setkomafont{section}{\normalsize}						% Schrift f�r Section
\setkomafont{subsection}{\normalsize}					% Schrift f�r Subsection

%-----------------------------------------------------------------------------
% "Quellcode"-Unterschrift von Listing in Quellcode umbennen
%-----------------------------------------------------------------------------
\addto{\captionsngerman}{\renewcommand*{\lstlistingname}{Quellcode}}

%-----------------------------------------------------------------------------
% Farbe f�r Links in PDF-Dokumenten definieren
%-----------------------------------------------------------------------------
\definecolor{LinkColor}{rgb}{0,0,0}				% Festlegen einer neuen Farbe

\hypersetup{colorlinks=true,					% farbliche Links
						breaklinks=true,		% Zeilenumbruch erlauben
						linkcolor=black,		% Farbe f�r interne Links
						citecolor=black,		% Farbe f�r Links zum Literaturverzeichnis
						filecolor=LinkColor,	% Farbe f�r externe Dateilinks
						menucolor=LinkColor,	%
						urlcolor=LinkColor}		% Farbe f�r externe Links
						
%-----------------------------------------------------------------------------
% Definition f�r Quelltextlistings
%-----------------------------------------------------------------------------
\lstloadlanguages{C}

\definecolor{lbcolor}{gray}{0.95}			% Farbe f�r den Hintergrund definieren				
\definecolor{darkblue}{rgb}{0,0,.6}		% Farbe f�r Schl�sselw�rter
\definecolor{darkred}{rgb}{.6,0,0}		% Farbe f�r Strings
\definecolor{darkgreen}{rgb}{0,.6,0}		% Farbe f�r Kommentare

\lstset{language=C,								% Programmiersprache der Listings
				alsolanguage=Matlab,			% alternative Programmiersprache der Listings
				frame=none,								% keinen Rahmen
				frameround=ffff,					% wenn ein Rahmen dargestellt werden soll, sind die Ecken spitz
				captionpos=b,							% Position der Benennung
				numbers=left,							% Zeilennummern links angeben
				stepnumber=1,							% in welchem Abstand sollen Zeilennummern angeben werden (1 2 3..)
				numbersep=3pt,						% Abstand zwischen Nummerierung und Listing
				numberstyle=\tiny,				% gr�sse der Nummern
				breaklines=true,					% Zeilenumbruch zulassen
				breakautoindent=true,
				postbreak=\space,
				tabsize=4,								% Tabulator auf 4 setzen
				escapechar=\$,
				basicstyle=\scriptsize\ttfamily,
				keywordstyle=\color{darkblue}\bfseries\ttfamily,	% Darstellung der Schl�sselw�rter
				stringstyle=\ttfamily\color{darkred},  						% Darstellung der Strings
				commentstyle=\itshape\color{darkgreen},						% Darstellung der Kommentare
				showspaces=false,					% leerzeichen nicht anzeigen
				showstringspaces=false,		% keine Leerzeichen bei Strings anzeigen
				xleftmargin=.52cm,
				xrightmargin=.52cm,				
				backgroundcolor=\color{lbcolor}}	% Hintergrundfarbe des Listings
				
%-----------------------------------------------------------------------------
% Kopf- und Fusszeile bestimmen
%-----------------------------------------------------------------------------
\pagestyle{fancy}	
\fancyhf{}												% alle Felder l�schen
\fancypagestyle{plain}{}

% Kopfzeile rechts bzw. au�en
\fancyhead[R]{\nouppercase{\leftmark}}
% Linie oben
\renewcommand{\headrulewidth}{0.5pt}
% Fu�zeile rechts bzw. au�en
\fancyfoot[R]{\thepage}
%-----------------------------------------------------------------------------

%-----------------------------------------------------------------------------
% Begin des Dokuments
%-----------------------------------------------------------------------------
\begin{document} 						% Beginn des Dokumentes

	\renewcommand\lstlistingname{Code}
	\renewcommand\lstlistlistingname{Codeverzeichnis}
	
	%% Titel
	\begin{titlepage}
		\setlength\headsep{-5mm}
		\begin{figure}[!h]
			\begin{minipage}{0.8\textwidth}
				\textbf{Hochschule Wismar} \\
				University of Applied Sciences \\
				Technology, Business and Design \\
				Fakultät für Ingenieurwissenschaften, Bereich EuI \\
			\rule{\textwidth}{0.5pt}
			\end{minipage}
			\begin{minipage}[r]{0.1\textwidth}
				\begin{flushright}
					\includegraphics[height=6\baselineskip]{pictures/HS-Wismar_Logo-FIW_2010-01.jpg}
				\end{flushright}
			\end{minipage}
		\end{figure}
		\vspace*{6cm}
		\begin{center}
			\Huge
			\textbf{Projektarbeit} \\
			\vspace{2cm}
			\large \Titel
			\begin{table*}[b]
				\begin{tabular}{rl}
					Gedruckt am: & \today \\
					\\
					von: & \VerfasserA \\
					& geboren am \GeburtstagA \\
					& in \GeburtsortA \\
					\\
					von: & \VerfasserB \\
					& geboren am \GeburtstagB \\
					& in \GeburtsortB \\
					\\
					Betreuer: & \Betreuer \\

				\end{tabular}
			\end{table*}
		\end{center}
	\end{titlepage}

	\onehalfspacing 					% 1 1/2-zeilig (package 'setspace')
	
	%\blankpage	%leeres Blatt zwischen Deckblatt und Inhaltsverzeichnis	
	%-----------------------------------------------------------------------------
	% Inhaltsverzeichnis
	%-----------------------------------------------------------------------------	
	\pdfbookmark[1]{Inhaltsverzeichnis}{toc}	% Inhaltsverzeichnis zu den Lesezeichen hinzufügen
	%\singlespacing 						% 1-zeilig
	\tableofcontents					% Inhaltverzeichnis einf�gen
	%\onehalfspacing 					% 1 1/2-zeilig (package 'setspace')

	%-----------------------------------------------------------------------------
	% Hauptteil
	%-----------------------------------------------------------------------------

\newpage

\chapter{Einleitung}
Möchten Studierende wissen welche Module sie nächstes Semester belegen oder in welchem Haus sie einen bestimmten Professor finden können, müssen sie sich durch endlos viele Seiten der Hochschul-Website klicken, um an das Ziel zu gelangen. Die Website ist sowohl für zukünftige Studierende und Interessierte, als auch für eingeschriebene Studierende konzipiert. Daher gibt es eine Fülle von Informationen und Querverweise, sodass es schwer ist die wichtigen Aspekte herauszufiltern.

Um dies zu vereinfachen wird im folgenden eine Applikation entwickelt die alle wichtigen Informationen zum Studiengang Angewandte Informatik - Multimendiatechnik zusammenfasst und gebündelt darstellt. Diese kann kann von allen Studierenden des Studienganges bequem auf dem Smartphone ausgeführt werden.

Nach einer Analyse der Hochschulseite und der Beschreibung des IST-Zustandes folgt die Festlegung der Systemanforderungen an die Applikation. In den Grundlagen wird erläutern was Webcomponents, im speziellen die LitElements, sind und wie sie aufgebaut sind. Hier wird zudem auf den Shadow-Dom, die Templates und die Properties eingegangen. Außerdem wird die interne Navigation mittels eines Routers beschrieben. Im Abschnitt Konzept wird der Aufbau und die Funktionsweise der Applikation näher erläutert. Im Kapitel 5 Implementierungen werden wichtige Ausschnitte aus dem Programmcode gezeigt und näher erklärt.

\chapter{Untersuchung der Hochschul-Website}	
Studenten die Informationen zu ihrem Studiengang suchen brauchen teilweise sechs Klicks um von der Startseite der HS Wismar zur Semsterübersicht (AIMMT) zu gelangen. Innerhalb dieser Seiten gibt es einige Unstimmigkeiten und fehlerhaftes Verhalten, welches im folgendem näher erläutert wird.

		%----------------------------------------------------------------------		
		\section{Analyse des Ist-Zustandes}
		Die Webseite der Hochschule hat einen großen allgemeinen Teil der Informationen über die Hochschule enthält und Verlinkungen zu den drei Fakultätsseiten. Da die entwickelte Webanwendung für Studierende des Studiengangs Angewandte Informatik und Multimendiatechnik gedacht ist wird an dieser Stelle nur die Fakultätsseite der Ingenieurstechnik bzw. des Bereiches Elektrotechnik und Information untersucht.
		
		In der oberen Navigationsleiste gibt es nicht eindeutig erkennbare Icons für den Schnelleinstieg und Informationen. Werden diese angeklickt, klappt sich ein Panel nach oben aus. Die obere Navigationsleiste ist nicht fixiert. Der Footer ist sehr groß und enthält teilweise die selben Verlinkungen wie in der Seitennavigation oben. Auf der Seite der Semsterübersicht gib es ein Akkordeonmenü welches nicht an allen Punkten anklickbar ist. Dieses Menü enthält ein Pfeil-Icon nach unten zeigend, welches dem User suggeriert das sich hinter dem Icon mehr Informationen enthalten.  Dieses Icon ist aber nicht anklickbar. Die Pfeil-Icons gibt es auf der gesamten Website als wiederkehrendes Symbol für Verlinkungen. Auf den Informationsseiten werden diese Icons allerdings auch als Auflistungszeichen verwendet. Weiterhin gibt es viele Querverweise. Auf einer Modulübersichtsseite befinden sich Verlinkungen zu weiteren Studiengängen die dieses Modul besuchen, weitere Module die der/die Professor|in unterrichtet, die Forschungsthemen, Thesenthemen, und Jobs die der/die Professor|in anbietet. Diese Informationen gibt als sowohl auf der Modulübersicht als auch auf der persönlichen Seite der/des Professor|in. Der User wird überfordert, da zu viele Informationen gegeben werden. Des weiteren fehlt die Auflistung der Wahlpflichtmodule. Diese müssen umständlich im Modulhandbuch gesucht werden. 

%----------------------------------------------------------------------		
		\section{Festlegung Systemanforderung}
		Die Website der Hochschule ist sowohl für Studieninterssierte, als auch für eingeschriebene Studierende konzipiert. Um schneller an die wichtigsten Informationen zu gelangen, soll die Anwendung nur für eingeschriebene Studierende des Studiengangs Angewandte Informatik und Multimediatechnik dienen. Weiterhin sollen nur die wichtigsten Information kurz und knapp dargestellt werden. Daher hat jede Seite ein klares Ziel.	So ergeben sich weniger Verlinkungen, sodass der User immer genau weiß wo er sich befindet. Die heutigen Studierenden gehören zur Generation Smartphone. Daher soll der Ansatz "mobile first" verfolgt werden.
	
%___________________________________________________________________________________
\chapter{Grundlagen}		
Um die folgenden Kapitel besser einzuordnen, werden nun einige technische Grundlagen betrachten. Zuerst wird auf die Webcomponents und im speziellen auf die LitElemente und deren Prinzipien eingegangen, danach wird auf die Navigation mittels eines Routers.
	
		\section{Web Components}
		Web Components stellen eine Reihe von Webplattform-APIs dar. Mit Ihnen kann wiederverwendbares gekapseltes HTML erstellt und erweitert werden. Das HTML wird in Komponenten gegliedert, die in jedem modernen Browser, jeder JavaScript-Bibliothek und jedem Framework verwendet werden kann. Web Components basieren auf vier Anforderungen: 
\begin{itemize}
\item \textit{Custom Elements:} Grundlage für das Erstellen neuer DOM-Elemente
\item \textit{Shadow Dom:} Definiert wie gekapseltes HTML in den Components verwendet wird
\item \textit{ES Modules:} Definiert die Einbindung und Wiederverwendung der JavaScript-Dateien
\item \textit{HTML Templates:} Definiert HTML-Fragmente die erst zur Laufzeit instanziiert werden
\end{itemize} Es existieren viele Bibliotheken mit denen die Erstellung von Web Components erleichtert wird, wie zum Beispiel Hybrids, LitElement, Polymer usw. Im folgenden wird die Bibliothek LitElement näher erläutert.\cite{webcom}

%----------------------------------------------------------------------				
				\subsection{LitElement}
				LitElement ist eine Basisklasse zum Erstellen von Web Components. Sie verwendet lit-html um Templates zu definieren und zu rendern und fügt eine API zum Verwalten von Eigenschaften und Attributen hinzu.  Die Eigenschaften werden beobachtet und die Elemente werden asynchron aktualisiert, wenn sich ihre Eigenschaft ändert. 

%----------------------------------------------------------------------					
					\subsection{Shadow-Dom}
					Der Shadow-Dom wird verwendet um den Template-Dom zu Kapseln. Er bietet drei wesentliche Vorteile: 
					\begin{itemize}
						\item \textit{Dom-Scoping:} DOM-APIs finden keine Elemente im Schatten-Dom, sodass globale Scripte keinen Zugriff haben.
						\item \textit{Style-Scoping:} Die gekapselte Styles haben keine Auswirkung auf den Rest des DOM-Baumes.
						\item \textit{Composition:} Der Schatten-Dom der Komponente ist von untergeordneten Elementen getrennt, so kann gesteuert werden wie untergeordnete Elemente in das Template gerendert werden sollen.\cite{webcom}
					\end{itemize}
														%----------------------------------------------------------------------
					\subsection{Templates}
					In einer Render-Funktion der Elementklasse wird das Template für die Component definiert. In dieser Funktion wird das rohe HTML in einem JavaScript template literal innerhalb von back-ticks geschrieben (siehe Abb. ). Die Render Methode kann alles zurückgeben was lit-html rendern kann. \cite{litelem}
					\begin{figure}[h]
						\centering
						\includegraphics[width=1\textwidth]{pictures/render-function}
						\caption{Render-Funktion}\cite{litelem}
						\label{Render-Funktion}
					\end{figure}
				
					\subsection{Properties}
					LitElement verwaltet deklarierte Eigenschaften in einem statischen property Getter, die entsprechenden Attribute werden in einem Elementkonstruktor initialisiert. So kann sicher gestellt werden das sich bei einer geplanten Elementaktualiserung die deklarierte Eigenschaft ändert.  \cite{litelem}
		
		%----------------------------------------------------------------------
		\section{Router}
	Der Router übernimmt das Navigieren auf der gesamten Seite. Er legt die angezeigte URL fest und verwaltet welche Ansicht gezeigt werden soll. Die verschiedenen Routen werden mit Namen und einem URL Pattern angelegt und können so auseinander gehalten werden. Über Parameter und Querys können dann Informationen übertragen werden, die für die neu aufgerufene Seite von belangen sind.\\
\\
Wird ein Button mit einem Redirect auf eine neue Ansicht belegt, registriert auch der Router den Klick und wechselt die Path Variable auf den entsprechenden Namen, welcher zu der neuen URL zugehörig ist. Anhand dieser Variable können alle nicht benötigten Ansichten ausgeblendet und die neuen angezeigt werden.				

%___________________________________________________________________________________
\chapter{Konzept}	
Anhand der in Kapitel 2 festgelegten Anforderungen	wird im folgenden ein Konzept in Form von einem Zustandsdiagramm und die Aufteilung in Komponenten entworfen.

%		\section{Use-Cases der Anwendung}	
%		Das in Abbilding 4.1 gezeigte Use-Case-Diagram beinhalten zu Beginn zwei mögliche Interaktionen. Der User kann zwischen der Übersicht der Semester oder der Übersicht der Professoren und Mitarbeiter wählen. Wählt der User die Semester Übersicht kann er nun ein Semester wählen und bekommt anschließend die Module des jeweiligen Semesters angezeigt. Entscheidet er sich hier für ein Modul werden ihm die Details des Moduls gezeigt. In der Professoren und Mitarbeiter Übersicht kann der User eine Person auswählen und bekommt dann ebenfalls die Details zu dieser Person gezeigt.
%	\begin{figure}[h]
%		\centering
%		\includegraphics[width=1\textwidth]{pictures/use-cases.png}
%		\caption{Use-Case-Diagramm (eigene Darstellung)}							
%		\label{Use-Case-Diagramm}
%	\end{figure}	
		
		%----------------------------------------------------------------------
		\section{Systementwurf}
		Die Abbildung 4.1 zeigt die Zustände, in denen sich der User während der Benutzung der Anwendung befinden kann. Im Zustand der Startseite kann durch ein Klick auf die gewünschte Übersicht zum einen in den Zustand der Semester Übersicht und zum anderen in die Übersicht der Professoren und Mitarbeiter gewechselt werden. Diesen Zustandswechsel der beiden Übersichten kann jederzeit durch die untere Hauptnavigation getätigt werden. Im Zustand der Semester Übersicht kann durch ein Klick auf ein Semester in den Zustand der Modul Übersicht gewechselt werden. Wählt der User hier durch ein Klick ein Modul aus, wird die entsprechende Detailseite aufgerufen. Wird im sechsten Semester das Wahlpflichtmodul gewählt, wird wieder ein Übersichtszustand aufgerufen, von dem ebenfalls in die Detailansicht gewechselt werden kann. Im Zustand der Professoren und Mitarbeiter Übersicht kann durch einen Klick auf eine Person die entsprechende Detailseite der Person aufgerufen werden.			
		\begin{figure}[h]
			\centering
			\includegraphics[width=0.7\textwidth]{pictures/zustandsdiagram.png}
			\caption{Zustandsdiagramm (eigene Darstellung)}						
			\label{Zustandsdiagramm}
		\end{figure}
\newpage					
		%----------------------------------------------------------------------
		\section{Ansichten und Komponenten}	
		 Die Webapplikation wurde in sechs Ansichten aufgeteilt: eine Startseite, die Semesterübersicht, die Modulübersicht, die Detailseite der Module, die Professorenübersicht und die Detailseite der Professoren. Aufgrund der verschiedenen Wahlpflichtmodule gibt es noch eine siebte Ansicht dieser Module. Innerhalb dieser Ansichten gibt es drei Komponenten die in allen wieder verwendet werden: die Überschrifts-Komponente, die Return-Button-Komponente und die Hauptnavigations-Komponente. Der Hauptteil der Ansichten ist in weitere Komponenten gegliedert:
	\begin{itemize}
		\item Auflistung der Semester
		\item Auflistung der Professoren und Mitarbeiter
		\item Auflistung der Module
		\item Semester-Wochen-Stundenübersicht
		\item Prüfungsinformationen
		\item Inhalt der Module
		\item Bild des Professors bzw. des Mitarbeiters
		\item Kontaktdaten des Professors bzw. des Mitarbeiters
		\item Lehre des Professors bzw. des Mitarbeiters
	\end{itemize}	
Diese wiederverwendbaren Komponenten können mit dem speziellen Content des jeweilig ausgewählten Semesters, Moduls oder einer bestimmten Person gefüllt werden. Dieser Content wird in einer erweiterbaren JSON-Datei gebündelt (siehe Kapitel 5.2 Datenhandling).



%___________________________________________________________________________________	
\chapter{Implementierungen}  
Im folgendem werden auf die für die Anwendung spezifischen Implementieren eingegangen, die auf Basis des im vorherigem Kapitels entworfenen Konzepte beruhen.

	\section{Komponenten}
	Alle Ansichten legen die Struktur der Webseite fest, indem sie die jeweiligen Komponenten durch dessen Custom Element (definiertes HTML-Tag mit eigenem Verhalten) einbinden (siehe Abb. 5.1 Ansicht der Details eines Semesters). 
	
	\begin{figure}[h]
		\centering
		\includegraphics[width=1\textwidth]{pictures/custom-element.png}
		\caption{Ansicht der Details eines Semesters}						
		\label{custom-element}
	\end{figure}
	
	Das Verhalten des jeweiligen Custom Elements wird über ein HTML-Template innerhalb der Render-Funktion implementiert. Dieses besteht aus herkömmlichen HTML-Tags (siehe Abb. 5.2 HTML-Template eines Custom Elements).

	\begin{figure}[h]
		\centering
		\includegraphics[width=1\textwidth]{pictures/html-template.png}
		\caption{HTML-Template eines Custom Elements}						
		\label{html-template}
	\end{figure}
\newpage	
	%----------------------------------------------------------------------
	\section{Datenhandling}
	Wie in Abbildung 5.3 zu sehen ist, besteht die Struktur der Daten aus JavaScript-Objekten. Das Objekt \textit{dataSem} enthält ein Array das wiederum Objekte der einzelnen Semester beinhaltet. Dieses besteht wieder aus einem Array mit Objekten der einzelnen Module.  Das Objekt \textit{dataProf} enthält ein Objekt Professoren und eine Objekt Mitarbeiter. Diese enthalten wieder ein Array aus Objekten mit den Daten der einzelnen Personen.
	
	\begin{figure}[h]
   		\begin{minipage}[b]{.4\linewidth} % [b] => Ausrichtung an \caption
      		\includegraphics[width=\linewidth]{pictures/dataSem.png}
   		\end{minipage}
   		\hspace{.1\linewidth}% Abstand zwischen Bilder
   		\begin{minipage}[b]{.4\linewidth} % [b] => Ausrichtung an \caption
      		\includegraphics[width=\linewidth]{pictures/dataProf.png}	
   		\end{minipage}
   		\caption{Objektstruktur der Semester und Module und der Professoren und Mitarbeiter}
   		\label{datastructure}
\end{figure}
	
	Durch spezielle Getter-Methoden werden die jeweiligen Daten aus den Objekten in die Komponenten geschrieben. In der Abbildung 5.4 Methode für das Datenhandling wird am Beispiel \textit{getTeaching} gezeigt wie die Daten für die Komponente \textit{Lehre} aus der \textit{dataProf} gezogen werden. Hierzu wird die Professoren ID der angeklickten Person aus der Route mit der ID in der \textit{dataProf} verglichen und die benötigten Daten in ein Array geschrieben. Dieses Array wird dann, wie in Abbildung 5.2 HTML-Template eines Custom Elements gezeigt, mithilfe der Map-Funktion in das Template geschrieben.
	\begin{figure}[h]
		\centering
		\includegraphics[width=1\textwidth]{pictures/getTeaching.png}
		\caption{Methode für das Datenhandling}						
		\label{getTeaching}
	\end{figure}
\newpage	
	%----------------------------------------------------------------------
	\section{Navigation mithilfe des Routers}
	Jedes anklickbare Element enthält einen Clickhandler, welcher die Navigation zur nächsten Seite ermöglicht und über den Router gesteuert wird. Welche Daten benötigt sind kann über die URL herausgefunden und in der speziellen Getter-Methode abgefragt werden.	\\
\\	
	Wie in Abbildung 5.5 zu sehen, werden in der \textit{routes()}-Methode alle benötigten Routen festgelegt.
Der Name gibt an welches Ziel die die Route hat. Während das Pattern festlegt wie die zugehörige URL aussieht. Diese wird mit Variablen versehen (\textit{semesterID}), um beispielsweise die verschiedenen Semester auseinander zu halten. Parameter werden nicht in der URL angezeigt, sondern direkt an die einzelnen Ansichten übergeben, um die Information weiter zu verarbeiten. 
\begin{figure}[h]
		\centering
		\includegraphics[width=1\textwidth]{pictures/routes-methode.png}
		\caption{routes()-Methode}						
		\label{routes}
	\end{figure} \\
	\\
	In der Render-Funktion (siehe Abb. 5.6) werden alle Ansichten über ihre Tags eingebunden und anhand der 'route' Variable, welche den Namen der aktuell aufgerufenen Seite enthält, ein und ausgeblendet.\\	
	\begin{figure}[h]
		\centering
		\includegraphics[width=1\textwidth]{pictures/router-render.png}
		\caption{Render-Methode des Routers}						
		\label{routes}
	\end{figure}
\\
\\
%___________________________________________________________________________________
\chapter{Zusammenfassung und Ausblick}     
Innerhalb dieses Projektes wurde eine Anwendung entwickelt die es Studenten erleichtert an wichtige Informationen,   während ihres Studiums der Angewandten Informatik an der Hochschule Wismar, zu gelangen. Dazu wurde als Erstes eine Analyse der Hochschul-Website vorgenommen und anhand der Ergebnisse Anforderungen an die Webapplikation festgelegt. Weiterhin wurden Web Components allgemein und im speziellen die LitElemente erläutert. Im Kapitel 4 Konzepte wurde der Systementwurf und die Aufteilung der Anwendung in die einzelnen Komponenten und Ansichten vorgestellt. \\
\\
Die entwickelte Anwendung erfüllt die in Kapitel 2 geforderten Kriterien. Es wurde der "mobile first" Ansatz verfolgt und in den Developer-Tools des Google Chrome-Browsers getestet (siehe Anhang A). Die WebApp stellt alle sieben Semester mit den jeweiligen Modulen und deren Details, sowie alle relevanten Professoren und Mitarbeitern dar. Alle Informationen sind einfach und schnell in den JSON-Dateien aktualisierbar.\\
\\
In weiteren Schritten ist angedacht die Master-Semester mit aufzunehmen, da bisher nur die Bachelor-Semester dargestellt sind. Weiterhin wird die Breadcrump-Komponente mit aufgenommen. So hat der User eine bessere Orientierung in welchem Teil der Anwendung er sich befindet.
		
	
 	%-----------------------------------------------------------------------------
	% Literaturverzeichnis einfügen, 
	% Nutzung der BibTeX-Technologie --> literatur.bib 
	%-----------------------------------------------------------------------------
	
%	\bibliographystyle{unsrtdin}		%  Stil des Literaturverzeichnisses (hier nach DIN 1505)
%	\bibliography{literatur.bib}			% gibt Datei mit der Literatur an
	\nocite{*}						% damit alle in der DB enthaltende Einträge bearbeitet werden


\begin{thebibliography}{unsrt}
\bibitem{webcom} \textit{Introduction - What are web componts?} \\
https://www.webcomponents.org/introduction [17.05.2020]

\bibitem{litelem} \textit{LitElement} 2018 Polymer Project\\
https://lit-element.polymer-project.org/ [17.05.2020]

\bibitem{router} \textit{LitElement Router}\\
https://www.npmjs.com/package/lit-element-router [28.05.2020]

\end{thebibliography}	

	%-----------------------------------------------------------------------------
	% Verzeichnisse
	%-----------------------------------------------------------------------------
	\listoffigures						% Bildverzeichnis einfügen
%	\listoftables						% Tabellenverzeichnis einfügen
%	\lstlistoflistings					% Quellcodeverzeichnis einfügen

	%-----------------------------------------------------------------------------
	% Anhang
	%-----------------------------------------------------------------------------	
	\appendix
	% Auch hier sind Gliederungen aller \chapter, \section
	\newpage
\section*{Anhang}
\setcounter{figure}{0}
\renewcommand{\thefigure}{\Alph{figure}}
\begin{figure}[h]
	\centering
	\includegraphics[width=1\textwidth]{pictures/anhang}
	\caption{Screenshots der Ansichten} 
	\label{filtercode}
\end{figure}


	%-----------------------------------------------------------------------------
	% Selbstständigkeitserklärung
	%-----------------------------------------------------------------------------	
	\chapter*{Selbstständigkeitserklärung}
	\addcontentsline{toc}{chapter}{Selbstständigkeitserklärung}
	\rhead{Selbstständigkeitserklärung} % rechts oben in der Kopfzeile Chapter darstellen
	Hiermit erklären wir, dass wir die hier vorliegende Arbeit selbstständig,
	ohne unerlaubte fremde Hilfe und nur unter Verwendung der aufgeführten
	Hilfsmittel angefertigt haben.

	\begin{tabular}{p{6cm}p{7cm}}
		\\
  		\\
  		\\
  		\\
  		Ort, Datum & Unterschriften
	\end{tabular}
	
	
	
\end{document}							% Ende des Dokuments
%-----------------------------------------------------------------------------
